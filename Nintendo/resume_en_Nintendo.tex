%-------------------------
% Resume in Latex
% Author : Qian Bao
% License : MIT
%------------------------

\documentclass[letterpaper,11pt]{article}

\usepackage{latexsym}
\usepackage[empty]{fullpage}
\usepackage{titlesec}
\usepackage{marvosym}
\usepackage[usenames,dvipsnames]{color}
\usepackage{verbatim}
\usepackage{enumitem}
\usepackage[hidelinks]{hyperref}
\usepackage{fancyhdr}
\usepackage[english]{babel}

% TIPA is a system for processing phonetic symbols in latex. 
\usepackage{tipa}
% Heart of xelatex. In order to utilize ttf and ttc font.
\usepackage{fontspec}

\pagestyle{fancy}
\fancyhf{} % clear all header and footer fields
\fancyfoot{}
\renewcommand{\headrulewidth}{0pt}
\renewcommand{\footrulewidth}{0pt}

% Adjust margins
\addtolength{\oddsidemargin}{-0.5in}
\addtolength{\evensidemargin}{-0.5in}
\addtolength{\textwidth}{1in}
\addtolength{\topmargin}{-.5in}
\addtolength{\textheight}{1.0in}

\urlstyle{same}

\raggedbottom
\raggedright
\setlength{\tabcolsep}{0in}

% Sections formatting
\titleformat{\section}{
  \vspace{-4pt}\scshape\raggedright\large
}{}{0em}{}[\color{black}\titlerule \vspace{-5pt}]

%-------------------------
% Custom commands
%------------------------
\newcommand{\resumeItem}[2]{
  \item\small{
    \textbf{#1}{: #2 \vspace{-2pt}}
  }
}

\newcommand{\resumeSubheading}[4]{
  \vspace{-1pt}\item
    \begin{tabular*}{0.97\textwidth}[t]{l@{\extracolsep{\fill}}r}
      \textbf{#1} & #2 \\
      \textit{\small#3} & \textit{\small #4} \\
    \end{tabular*}\vspace{-5pt}
}

\newcommand{\resumeSubItem}[2]{\resumeItem{#1}{#2}\vspace{-4pt}}

\renewcommand{\labelitemii}{$\circ$}

\newcommand{\resumeSubHeadingListStart}{\begin{itemize}[leftmargin=*]}
\newcommand{\resumeSubHeadingListEnd}{\end{itemize}}
\newcommand{\resumeItemListStart}{\begin{itemize}}
\newcommand{\resumeItemListEnd}{\end{itemize}\vspace{-5pt}}

%------------------------
%  CV STARTS HERE
%------------------------

\begin{document}

%---------HEADING--------
\begin{tabular*}{\textwidth}{l@{\extracolsep{\fill}}r}
  \textbf{\href{https://github.com/houkensjtu}{\Large Qian Bao}} & Email : \href{mailto:houkensjtu@gmail.com}{houkensjtu@gmail.com}\\
   & Mobile : +1-610-704-4131 \\
\end{tabular*}


%-----------SKILLS-----------------
\section{Skills}
  \resumeSubHeadingListStart
     \resumeSubItem{Programming}
     {Self learned a broad range of undergraduate level computer science courses including but not limited to:
      \begin{itemize}
        \item The Hardware/Software interface (C)
        \item Data structure and algorithms (Java/Python)
        \item Principles of imperative computation (C) and object-oriented programming (Python)
        \item Fundamental of artificial intelligence (Matlab/Python)
      \end{itemize}
     10+ years experience in Linux operating system and have familiarity with shell / Git / script languages / compilers.
     Also keen on building personal projects including games(using Pygame), toy computer emulator and a 4 bit computer from scratch.}

    \resumeSubItem{Product development}
      {5+ years experience in a next-generation product development team. 
      Carried out numerical analysis (using Python, Fortran and opensource C++ code) to assist product design and accelerated development cycle. 
      Designed some of the key components of the product and also wrote internal technical procedures and manuals for co-workers.}

    \resumeSubItem{Foreign language/communication}
      {5 years working experience in Japan and 1 year in the US. Worked well with co-workers from diversed cultural background and
      built positive human relationship. Presented multiple times at international (in English) and domestic conferences (in Japanese).}

    \resumeSubHeadingListEnd

%----------Language skill-------
\section{Language Certification}
\resumeSubHeadingListStart
  \resumeSubItem{Mandarin Chinese}{Mother tongue}
  \resumeSubItem{English}{Professional efficiency; TOEFL iBT 108/120; GRE V156 + Q170; TOEIC 965/990}
  \resumeSubItem{Japanese}{Professional efficiency; JLPT N1 176/180}
  \resumeSubItem{German}{Intermediate efficiency; CEFR B1}
\resumeSubHeadingListEnd

%-----EXPERIENCE----------
\section{Experience}
  \resumeSubHeadingListStart

    \resumeSubheading
    {Sumitomo Cryogenics of America, Inc}{Allentown, PA}
    {Research Engineer, Project Leader}{Oct 2018 - Present}
    \resumeItemListStart
      \resumeItem{Development of pneumatic drive high capacity Gifford-McMahon cryocooler}
        {  \begin{itemize}
           \item {Leading a cross-functional team operating the R\&D of a next-generation cryocooler product. }
           \item {Facilitated the communication among design team, business stakeholders, global partners and headquarters.}
           \end{itemize}
        }
    \resumeItemListEnd

    \resumeSubheading
      {Sumitomo Heavy Industries, Ltd.}{Tokyo, Japan}
      {Research Engineer}{Apr 2013 - Sep 2018}
      \resumeItemListStart
        \resumeItem{Development of next generation 2K Gifford-McMahon cryocooler}
        {  
           \begin{itemize}
           \item{Carried out numerical analysis of cryocooler products and designed the key components (regenerator etc.) of a new, highly compact Gifford-McMahon cryocooler
           which can achieve 2K on its second stage. Also built customized measurement system (GPIB) to test the prototype unit.}
           \item{Based on this work, published 2 papers in international cryogenic conference and 1 journal paper in Japan. 
           The design result is recognized as the most compact 2K Gifford-McMahon cryocooler in the world.}
           \item {This project was supported by the National Institute of Information and Communications Technology (NICT), JAPAN.}
          \end{itemize}
        }
      \resumeItemListEnd

  \resumeSubHeadingListEnd
  

%------EDUCATION---------
\section{Education}
  \resumeSubHeadingListStart
    \resumeSubheading
      {Tohoku University}{Sendai, Japan}
      {Master of Engineering in Graduate School of Mechanical Engineering;  GPA: 3.7/4.0}{Oct. 2010 -- Mar. 2013}
      
    \resumeSubheading
      {Shanghai Jiaotong University}{Shanghai, China}
      {Bachelor of Engineering in School of Mechanical Engineering;  GPA: 3.5/4.0}{Sep. 2006 -- Jun. 2010}
    \resumeSubheading
      {Shanghai Jiaotong University}{Shanghai, China}
      {Bachelor of Arts in School of Foreign Language;  Minor in Japanese language}{Sep. 2008 -- Jun. 2010}

  \resumeSubHeadingListEnd

%----------Publications-------
\section{Publication}
1.Experimental Investigation of Compact 2 K GM Cryocoolers, Physics Procedia, Volume 67, Pages 428-433, ISSN 1875-3892,
https://doi.org/10.1016/j.phpro.2015.06.053., (25th International Cryogenic Engineering Conference and the International Cryogenic Materials Conference in 2014, ICEC 25–ICMC 2014, Netherland)

2.Development of a compact 2K GM cryocooler, Cryogenic engineering (Japanese), Vol.50 (2015) No.3 p. 125-131

3.Recent development status of compact 2 K GM cryocoolers,
IOP Conference Series: Materials Science and Engineering,
volume 101,
p.012136.
(Cryogenic Engineering Conference ,CEC 2015: Tucson, AZ, USA, June 28 - July 2, 2015)

4.Development status of a high cooling capacity single stage GM cryocooler,
Cryocooler 19,
pp. 291-297,
(International Conference of cryocoolers ,ICC19: San Diego, CA, USA, June 20-23, 2016)

5.Development of a pneumatic GM cryocooler with dual-displacer (Oral presentation, International workshop on cooling of HTS-Applications, IWC-HTS, Karlsruhe, Germany, September 13-15, 2017)

%----------Patent-------------
\section{Patent}
1. A. T. A. M de Waele, Mingyao Xu and Qian Bao. Method to achieve temperature lower than 2K in a Gifford-McMahon cryocooler. U.S. Patent 10,197,305. Filed Dec 21,2015. Issued Feb 5,2019.

2. Mingyao Xu and Qian Bao. Optimization of concentric heat exchanger for cryogenic refrigerator. U.S. Patent 9,841,212. Filed Apr 2,2015. Issued Dec 12,2017.

3. Mingyao Xu, Takaaki Morie and Qian Bao. A high-efficiency dual-cylinder design of Gifford-McMahon cryocooler. U.S. Patent 10,184,693. Filed Oct 20,2016. Issued Jan 22,2019.

\end{document}
