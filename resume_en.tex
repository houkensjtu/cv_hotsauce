%-------------------------
% Resume in Latex
% Author : Qian Bao
% License : MIT
%------------------------

\documentclass[letterpaper,11pt]{article}

\usepackage{latexsym}
\usepackage[empty]{fullpage}
\usepackage{titlesec}
\usepackage{marvosym}
\usepackage[usenames,dvipsnames]{color}
\usepackage{verbatim}
\usepackage{enumitem}
\usepackage[hidelinks]{hyperref}
\usepackage{fancyhdr}
\usepackage[english]{babel}

% TIPA is a system for processing phonetic symbols in latex. 
\usepackage{tipa}
% Heart of xelatex. In order to utilize ttf and ttc font.
\usepackage{fontspec}

\pagestyle{fancy}
\fancyhf{} % clear all header and footer fields
\fancyfoot{}
\renewcommand{\headrulewidth}{0pt}
\renewcommand{\footrulewidth}{0pt}

% Adjust margins
\addtolength{\oddsidemargin}{-0.5in}
\addtolength{\evensidemargin}{-0.5in}
\addtolength{\textwidth}{1in}
\addtolength{\topmargin}{-.5in}
\addtolength{\textheight}{1.0in}

\urlstyle{same}

\raggedbottom
\raggedright
\setlength{\tabcolsep}{0in}

% Sections formatting
\titleformat{\section}{
  \vspace{-4pt}\scshape\raggedright\large
}{}{0em}{}[\color{black}\titlerule \vspace{-5pt}]

%-------------------------
% Custom commands
%------------------------
\newcommand{\resumeItem}[2]{
  \item\small{
    \textbf{#1}{: #2 \vspace{-2pt}}
  }
}

\newcommand{\resumeSubheading}[4]{
  \vspace{-1pt}\item
    \begin{tabular*}{0.97\textwidth}[t]{l@{\extracolsep{\fill}}r}
      \textbf{#1} & #2 \\
      \textit{\small#3} & \textit{\small #4} \\
    \end{tabular*}\vspace{-5pt}
}

\newcommand{\resumeSubItem}[2]{\resumeItem{#1}{#2}\vspace{-4pt}}

\renewcommand{\labelitemii}{$\circ$}

\newcommand{\resumeSubHeadingListStart}{\begin{itemize}[leftmargin=*]}
\newcommand{\resumeSubHeadingListEnd}{\end{itemize}}
\newcommand{\resumeItemListStart}{\begin{itemize}}
\newcommand{\resumeItemListEnd}{\end{itemize}\vspace{-5pt}}

%------------------------
%  CV STARTS HERE
%------------------------

\begin{document}

%---------HEADING--------
\begin{tabular*}{\textwidth}{l@{\extracolsep{\fill}}r}
  \textbf{\href{http://sourabhbajaj.com/}{\Large Qian Bao}} & Email : \href{mailto:houkensjtu@gmail.com}{houkensjtu@gmail.com}\\
  Address : 3273 W. Cedar Street, Allentown, 18104 PA & Mobile : +1-610-704-4131 \\
\end{tabular*}

%-----EXPERIENCE----------
\section{Experience}
  \resumeSubHeadingListStart

    \resumeSubheading
    {Sumitomo Cryogenics of America, Inc}{Allentown, PA}
    {Research Engineer, Project Leader}{Oct 2018 - Present}
    \resumeItemListStart
      \resumeItem{Development of pneumatic drive high capacity Gifford-McMahon cryocooler}
      {Leading the R\&D of a next-generation cryocooler product. Carried out numerical thermal/mechanical analysis of the key components (regenerator etc.)
      and proposed the prototyping strategies. At the same time, facilitated the communication among design team, business stakeholders, 
      global partners and headquarters.}
    \resumeItemListEnd

    \resumeSubheading
      {Sumitomo Heavy Industries, Ltd.}{Tokyo, Japan}
      {Research Engineer}{Apr 2013 - Sep 2018}
      \resumeItemListStart
        \resumeItem{Development of next generation 2K Gifford-McMahon cryocooler}
        {Carried out detailed analysis of previous cryocooler products and based on that, designed key components of a new, highly compact Gifford-McMahon cryocooler
        which can achieve 2K on its second stage. Based on this work, published 2 papers in international cryogenic conference and 1 journal paper in Japan. 
        The design result is recognized as the most compact 2K Gifford-McMahon cryocooler in the world.
        This project was supported by the National Institute of Information and Communications Technology (NICT), JAPAN.}
        \resumeItem{Development of next generation high capacity Gifford-McMahon cryocooler}
        {Carried out conceptual design and detailed analysis of a next generation high capacity Gifford-McMahon cryocooler. Built test facilities to assist the
        prototyping of innovative design concepts. Published 2 papers in international cryogenic conference based on this work.}
      \resumeItemListEnd

  \resumeSubHeadingListEnd
  

%------EDUCATION---------
\section{Education}
  \resumeSubHeadingListStart
    \resumeSubheading
      {Tohoku University}{Sendai, Japan}
      {Master of Engineering in Graduate School of Mechanical Engineering;  GPA: 3.7/4.0}{Oct. 2010 -- Mar. 2013}
      
      Thesis : Study of Heat Transfer and Fluid Flow of Bubbles in a microchannel; Advisor : Shigenao Maruyama

    \resumeSubheading
      {Shanghai Jiaotong University}{Shanghai, China}
      {Bachelor of Engineering in School of Mechanical Engineering;  GPA: 3.5/4.0}{Sep. 2006 -- Jun. 2010}
      
      Thesis : Mass flow distribution behavior of liquid nitrogen in a minichannel heat sink; Advisor : Peng Zhang
    \resumeSubheading
      {Shanghai Jiaotong University}{Shanghai, China}
      {Bachelor of Arts in School of Foreign Language;  Minor in Japanese language}{Sep. 2008 -- Jun. 2010}

  \resumeSubHeadingListEnd

  
%-----------SKILLS-----------------
\section{Skills}
  \resumeSubHeadingListStart
    \resumeSubItem{Mechnical design}
      {General knowledge of mechanical system, manufacturing process and design process. Can operate FEM (Ansys Mechanical)
      and CFD (Ansys Fluent, OpenFoam) analysis to facilitate and accelerate product prototyping. 3+ years experience of Solidworks. 
      CSWA certificated by Solidworks.}
    \resumeSubItem{Thermal/Fluid dynamic analysis}
      {Built numerical tools to assist product design (Python, Fortran, D3.js). Carried out thermal/fluid dynamic analysis (OpenFoam)
      of cryogenic heat exchanger to facilitate design improvement.}
    \resumeSubItem{Programming}
      {Broad knowledge of programming languages and general computer science as a mechanical engineer. Wrote numerical
      tools in C, Fortran, Python and Javascript to provide insight for design strategy making. 
      Built customized data acquisition software to assist in-house performance experiment.}
    \resumeSubItem{6-sigma (toolset of process improvement)}
      {General understanding of the 6-$\sigma$ toolset and process. 4+ years working as a team member and now leading the team
      going through the design review process of a new product as project leader.}
\resumeSubHeadingListEnd


%----------Language skill-------
\section{Language Skills}
\resumeSubHeadingListStart
  \resumeSubItem{Mandarin Chinese}{Mother tongue}
  \resumeSubItem{English}{Professional efficiency; TOEFL iBT 108/120; GRE V156 + Q170}
  \resumeSubItem{Japanese}{Professional efficiency; JLPT N1 176/180}
  \resumeSubItem{German}{Intermediate efficiency; CEFR B1}
\resumeSubHeadingListEnd


%----------Publications-------
\section{Publication}
1.Experimental Investigation of Compact 2 K GM Cryocoolers, Physics Procedia, Volume 67, Pages 428-433, ISSN 1875-3892,
https://doi.org/10.1016/j.phpro.2015.06.053., (25th International Cryogenic Engineering Conference and the International Cryogenic Materials Conference in 2014, ICEC 25–ICMC 2014, Netherland)

2.Development of a compact 2K GM cryocooler, Cryogenic engineering, Vol.50 (2015) No.3 p. 125-131

3.Recent development status of compact 2 K GM cryocoolers,
IOP Conference Series: Materials Science and Engineering,
volume 101,
p.012136.
(Cryogenic Engineering Conference ,CEC 2015: Tucson, AZ, USA, June 28 - July 2, 2015)

4.Development status of a high cooling capacity single stage GM cryocooler,
Cryocooler 19,
pp. 291-297,
(International Conference of cryocoolers ,ICC19: San Diego, CA, USA, June 20-23, 2016)

5.Development of a pneumatic GM cryocooler with dual-displacer (Oral presentation, International workshop on cooling of HTS-Applications, IWC-HTS, Karlsruhe, Germany, September 13-15, 2017)

%----------Patent-------------
\section{Patent}
1. A. T. A. M de Waele, Mingyao Xu and Qian Bao. Method to achieve temperature lower than 2K in a Gifford-McMahon cryocooler. U.S. Patent 10,197,305. Filed Dec 21,2015. Issued Feb 5,2019.   

2. Mingyao Xu and Qian Bao. Optimization of concentric heat exchanger for cryogenic refrigerator. U.S. Patent 9,841,212. Filed Apr 2,2015. Issued Dec 12,2017.   

3. Mingyao Xu, Takaaki Morie and Qian Bao. A high-efficiency dual-cylinder design of Gifford-McMahon cryocooler. U.S. Patent 10,184,693. Filed Oct 20,2016. Issued Jan 22,2019.   
\end{document}
